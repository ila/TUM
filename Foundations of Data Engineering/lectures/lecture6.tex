\section{NoSQL databases}
NoSQL databases are an alternative to traditional relational databases in which data is placed in tables and data schema is carefully designed before the database is built. 

Relational databases were seen as slow and complex, not suited for web usages, therefore NoSQL systems aim to provide a simpler and faster solution, horizontally scaled to clusters of machines.

NoSQL databases are especially useful for working with large sets of real-time distributed data, providing high throughput and availability. 

They do not provide ACID guarantees, relying instead on BASE principles: the trade off is a more limited query functionality, for instance many NoSQL do not support joining.

Giving up on consistency implies also giving up on additional protocols such as 2PC, but often compromises are taken: groups of entities that share affinities are co-located, providing transaction support within entity groups. 

To avoid data loss, write-ahead logging (WAL) saves changes to a log before applying them, allowing updates of a database to be done in-place.

\subsection{Key-value databases}
Key-value stores consist in associations between keys, usually primitive types, and values, which can be primitive or complex but are hardly ever directly accessed.

The API offers simple functionalities such as getting and settings, also working in multi-sets, but those do not provide consistency. 

Key-value stores make extensive use of hash tables to partition the space across multiple machines, relying on distributed hash tables to know which node to contact.

\subsection{MongoDB}
MongoDB is one of the most popular document-oriented database for JSON documents indexed by an unique identifier.

It provides high availability with replica sets, writing and reading on the primary by default. Scaling horizontally is achieved with sharding.

Aggregation can be performed in three ways: through a pipeline, MapReduce or aggregation operations, eventually attached together.

\subsection{Apache Cassandra}
Apache Cassandra is another NoSQL database system, built to be very scalable and available, sacrificing consistency. 

It provides atomicity at the partition level: inserting or updating columns are considered write operations, and not more than one node can perform one at the same time.

The system does not roll back if a write succeeds on one replica but fails in another, but it defines a QUORUM level to propagate changes and wait for acknowledgment. The lack of rollbacks guarantees atomicity.

Cassandra uses timestamps to determine the most recent update to a column, provided by the client application, and fetching the latest one when requesting data.

Consistency can be ensured in two ways:
\begin{itemize}
	\item Tunable consistency, giving up on availability according to CAP theorem but always keeping partition tolerance;
	\item Linearizable consistency, using a serial isolation level for lightweight transactions, performed when uninterrupted sequential scans are not dependent on concurrency.
\end{itemize}

Cassandra uses the Paxos consensus protocol, where all operations are quorum-based and updates will incur a performance hit. A consistency level of SERIAL has been added to support linearizable consistency.

Full row-level isolation is in place (but no more), which means that writes to a row are isolated to the client performing the write and are not visible to any other user until they are complete.

Data is partitioned, with partitions mapped to nodes in the cluster; scale out is achieved adding more nodes and redistributing data. 

Models can be efficiently stored through a partition key, controlling how data is spread, and a clustering key, deciding how data is sorted on disk.
