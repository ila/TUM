\documentclass[a4paper, english, headtopline=0.08em, headsepline=0.04em, left = 1cm, right = 1cm, DIV=15]{article}



\usepackage[left=1.5cm, right=1.5cm, top=2cm, bottom=3cm]{geometry}


\usepackage[main=english]{babel}
\usepackage[T1]{fontenc} % Schrift ordentlich rendern, damit Umlaute auch Umlaute sind
\usepackage[utf8]{luainputenc} % Interpretation des Inputs in utf8 (inkl. Umlaute, Akzente, etc.)
\usepackage{lmodern} % Ordentliche skalierbare Schriftart
\usepackage{blindtext}
\usepackage{dirtytalk}
\usepackage{slashed}
\usepackage{caption} %Abb. statt Abbildung bei Grafiken sowie Tab. statt Table
\usepackage{chngcntr}
\usepackage{xcolor}

\usepackage{framed}
\usepackage{mdframed}

\usepackage{graphicx} %zum einbinden von Grafiken
\usepackage{wrapfig} %zur wrapfigure-Umgebung, damit Grafiken nicht über die ganze Seitenbreite gehen
\usepackage{listings} %to write python code
\usepackage{multicol}
\usepackage{makeidx} 
\usepackage[all]{nowidow}
\usepackage{paralist}
\usepackage{perpage}
\usepackage{tikz}
    \usetikzlibrary{mindmap,decorations.pathreplacing,arrows,calc}
\usepackage[compat=1.1.0]{tikz-feynman}
\usepackage{pgfplots}
    \pgfplotsset{compat = 1.15}
    \def\sothreecirc{\path[thin,dashed,draw=black,fill=black!20] (0,0) circle (\rad)}
    \def\dotmiddle{\fill (0,0) circle (2pt)}
    \def\rad{1.8cm}
\usepackage{todonotes}

\usepackage[separate-uncertainty = true]{siunitx}
\usepackage{nicefrac} %zum darstellen von 1/2 in "schön"
\usepackage{upgreek}
\usepackage{cancel}
\usepackage{braket}
\usepackage{tensor}
\usepackage{marvosym}
\usepackage{mathtools}
\usepackage{mathrsfs} 
\usepackage{amsfonts, amsmath, amssymb} %Mathezeugs
\usepackage{fixmath}
\usepackage{bm}
\usepackage{dsfont}
\usepackage{minted} %to highlight code syntax

\usepackage{stackengine,scalerel}
\usepackage{calc}

\usepackage{hyperref}
\makeatletter
\def\@footnotecolor{myorange}
\define@key{Hyp}{footnotecolor}{%
 \HyColor@HyperrefColor{#1}\@footnotecolor%
}
\def\@footnotemark{%
    \leavevmode
    \ifhmode\edef\@x@sf{\the\spacefactor}\nobreak\fi
    \stepcounter{Hfootnote}%
    \global\let\Hy@saved@currentHref\@currentHref
    \hyper@makecurrent{Hfootnote}%
    \global\let\Hy@footnote@currentHref\@currentHref
    \global\let\@currentHref\Hy@saved@currentHref
    \hyper@linkstart{footnote}{\Hy@footnote@currentHref}%
    \@makefnmark
    \hyper@linkend
    \ifhmode\spacefactor\@x@sf\fi
    \relax
  }%
\makeatother
\hypersetup{
  linkcolor = blue,
  citecolor  = myorange,
  urlcolor   = myblue,
  colorlinks = true,
}
%
\definecolor{myred}  {HTML}{A3061E}
\definecolor{myblue} {RGB} {0,63,119}
\definecolor{myyellow} {cmy} {0,0.263,0.741}
\definecolor{mygreen} {HTML}{0B6E4F}
%
\colorlet{myorange} {myyellow!60!myred}
\colorlet{myviolett} {myred!50!myblue!80}
\renewcommand{\thefootnote}{\roman{footnote}}
\MakePerPage{footnote} %Fußnoten werden am Ende jeder Seite statt am Ende des Dokuments angezeigt


%\setcounter{tocdepth}{1} %Im Inhaltsverzeichnis werden Chapter und Sections angezeigt.

\newtheorem{theorem}{Theorem}[section]
\newtheorem{corollary}{Corollary}[theorem]
\newtheorem{lemma}[theorem]{Lemma}
\newtheorem{remark}{Remark}[section]

\newlength\shlength

\setlength{\parindent}{0pt}
\setlength{\parskip}{0.5\baselineskip}



\definecolor{shadecolor}{RGB}{192,192,192}

\DeclareMathOperator*{\argmax}{arg\,max}
\DeclareMathOperator*{\argmin}{arg\,min}
\newcommand{\ti}{\text{i}}
\newcommand{\td}{\text{d}}
\newcommand{\te}{\text{e}}
\newcommand{\prt}{\partial}
\newcommand{\hilb}{\mathcal{H}}
\newcommand{\CC}{\mathds{C}}
\newcommand{\RR}{\mathds{R}}
\newcommand{\NN}{\mathds{N}}
\newcommand{\epsi}{\varepsilon}
\newcommand{\epso}{\varepsilon_0}
\newcommand{\vphi}{\varphi}
\newcommand{\inte}{\int\limits}
\renewcommand{\(}{\left(}
\renewcommand{\)}{\right)}
\renewcommand{\[}{\left[}
\renewcommand{\]}{\right]}
\newcommand{\lb}{\left(}
\newcommand{\rb}{\right)}
\newcommand{\lsb}{\left[}
\newcommand{\rsb}{\right]}
\newcommand{\llb}{\left\lbrace}
\newcommand{\rrb}{\right\rbrace}
\newcommand{\llangle}{\left<}
\newcommand{\rrangle}{\right>}
\newcommand{\rra}{\rrangle}
\newcommand{\lla}{\llangle}
\newcommand{\labs}{\left|}
\newcommand{\rabs}{\right|}
\newcommand{\lno}{\left.}
\newcommand{\rno}{\right.}
\newcommand{\ub}[2]{\,\smash{\underbrace{#1}_{#2}}\vphantom{#1}\,}
\newcommand{\lt}{\leadsto}
\newcommand{\ra}{\rightarrow}
\newcommand{\Ra}{\Rightarrow}
\newcommand{\ua}{\uparrow}
\newcommand{\Ua}{\Uparrow}
\newcommand{\lra}{\leftrightarrow}
\newcommand{\Lra}{\Leftrightarrow}
\newcommand{\da}{\downarrow}
\newcommand{\Da}{\Downarrow}
\newcommand{\cra}{\curvearrowright}
\renewcommand{\braket}{\Braket}
\renewcommand{\bra}{\Bra}
\renewcommand{\ket}{\Ket}
\newcommand{\mel}{\matrixelement}
\renewcommand{\set}{\Set}
\renewcommand{\bar}{\overline}
\newcommand{\Pa}{\mathrm{d}}
\renewcommand{\mathbb}[1]{\mathds{#1}}
\newcommand{\pa}{\partial}
\newcommand{\defi}{\coloneqq} % :=
\newcommand{\Chi}{\mathcal{X}}
\newcommand{\dom}{\mathrm{dom}}
\newcommand{\mr}[1]{\mathrm{#1}}
\newcommand{\mf}[1]{\mathfrak{#1}}
\newcommand{\mc}[1]{\mathcal{#1}}
\newcommand{\mb}[1]{\mathbb{#1}}
\newcommand{\ms}[1]{\mathscr{#1}}
\newcommand{\tb}[1]{\textbf{#1}}
\newcommand{\tk}[1]{\textit{#1}}
\newcommand{\ul}[1]{\underline{#1}}
\renewcommand{\vec}[1]{\mathbold{#1}}
\newcommand{\define}{\coloneqq} % :=
\newcommand{\thor}{\Lightning{}}
\newcommand{\Hil}{\mathscr{H}}
\newcommand{\1}{\mathbbm{1}}
\newcommand{\Lag}{\mc{L}}
\newcommand{\E}{\mc{E}}
\newcommand{\Ham}{\mc{H}}
\newcommand{\Lap}{\pmb{\Delta}}
\newcommand{\tr}{\textbf{tr}}
\newcommand{\Var}{\textbf{Var}}
\newcommand{\sgn}{\textbf{sgn}}
\newcommand{\grad}{\textbf{grad}}
\newcommand{\rot}{\textbf{rot}}
\newcommand{\quabla}{\pmb{\square}}
\newcommand{\nab}{\pmb{\nabla}}
\renewcommand{\dfrac}[2]{\frac{\Pa #1}{\Pa #2}}
\newcommand{\delfrac}[2]{\frac{\pa #1}{\pa #2}}
\newcommand{\imp}[1]{\textbf{\color{blue!80!black}#1\color{black}}\\}
\newcommand{\mar}[1]{\textbf{#1}\\}
\newcommand{\J}{\mc{J}}
\newcommand{\RM}[1]{\MakeUppercase{\romannumeral #1{.}}}
\newcommand{\nbox}[1]{\fbox{\parbox{.5\textwidth}{#1}}}

\newcounter{lecturesc}

\newcommand{\lecture}[1]{{\refstepcounter{lecturesc} \par% Doppelte Klammern wichtig, damit graue Farbe eingeschränkt wird
  \color{gray}
  \bigskip%
  \hrule height 0.5pt%
  \kern 5pt%
  \hbox to \textwidth{\hfil\small\smash{End of lecture \thelecturesc\ on \textbf{#1}}\vphantom{M}\hfil}%
  \kern 5pt%
  \hrule height 0.5pt%
  \kern\medskipamount%
}} % Doppelte Klammern wichtig, damit graue Farbe eingeschränkt wird
%\renewcommand{\vorlesung}[1]{} %Einkommentieren, um Vorlesungsdaten auszublenden

\newenvironment{lastlecture}[1]{\begin{shaded*}
\begin{flushleft}
\textbf{Last Lecture: #1 }
\end{flushleft}
\begin{flushleft}}
{\end{flushleft}
\end{shaded*}}

\newmdenv[
  topline=false,
  bottomline=false,
  rightline = false,
  linewidth = 3pt,
  linecolor = gray,
  skipabove=\topsep,
  skipbelow=\topsep
]{siderules}

\newcommand{\arem}[1]{\begin{leftbar} #1 \end{leftbar}}
% template from https://latex.tum.de/project/618c4a584a8e454c5f4cfa1a
\begin{document}
{
    \begin{titlepage}
    	\centering
    	\vfill
    	{\scshape\LARGE Technische Universität München \par}
    	\vfill
    	{\scshape\Large Summary of the lecture MA4800\\   \par}
    	{\huge\bfseries Foundations in Data Analysis \par}
    	\vfill
    	{\Large\itshape Instructors: Dr. Anna Veselovska and  Felix Krahmer\par}
    	\vfill
    \end{titlepage}
}

\tableofcontents
\clearpage 

%%%%% section start %%%%%%%%%
\section{Linear Algebra Review}


\begin{itemize}
	\item We work on $\mathbb{K} \in\{\mathbb{R}, \mathbb{C}\}$.
	\item $A^H = \overline{(A^T)}$.
	\item A Hermitian matrix $A$ satisfies $A=A^H$.
	\item $A^{(i)}$ are rows and $A_{(j)}$ are the columns.
	\item $A^{(i)}=\left(a_{i j}\right)_{j \in J}$ and $A_{(j)}=\left(a_{i j}\right)_{i \in I}=\left(A^{T}\right)^{(j)}$
	\item The matrix-vector product between $A \in \mathbb{K}^{I \times J}$ and $x \in \mathbb{K}^{I}$ results in the vector in $A x \in K^{\prime}$ with entries

\end{itemize}

\subsection{Matrices}

\begin{align*}
	(A x)_{i}=\sum_{j \in J} a_{i j} x_{j} .
\end{align*}
\subsection{Matrix Multiplication}
The matrix-matrix product between $A \in \mathbb{K}^{I\times J}$ and $B \in \mathbb{K}^{J \times L}$ yields the matrix in $\mathbb{K}^{I \times L}$ with entries
\begin{align*}
	(A B)_{i \ell}=\sum_{j \in J} A_{i j} B_{j \ell} .
\end{align*}
%%%%% section end %%%%%%%%%


%%%%% section start %%%%%%%%%
\section{The Singular Value Decomposition}

\subsection{The Power Method}
\begin{lemma}
Let $x \in \mathbb{R}^{d}$ be a unit $d$-dimensional vector of components $x=$ $\left(x_{1}, \ldots, x_{d}\right)$ with respect to the canonical basis and picked uniformly at random from the sphere $\left\{x:\|x\|_{2}=1\right\}$. The probability that $\left|x_{1}\right| \geq \alpha>0$ is at least $1-C \alpha \sqrt{d}$ for some absolute constant.
\end{lemma}
\subsection*{Proof}
We want the probability of y picked uniformly at random from
\begin{align*}
	B^d(1)=\left\{y\in\RR^d,||y||_2 \leq 1\right\}
\end{align*}	
satisfies $|y_1|>\alpha$. In other words, we are looking for the fraction of $B^d(1)$ that satisfies $|y_1|>\alpha$.
This corresponds to
\begin{align*}
	V_{\alpha}:=\text{Vol}(B^d(1) \cap \{y : |y_1| \leq \alpha\})
\end{align*}
\begin{align*}
	=\int_{y\in B^d(1) \cap \{y : |y_1| \leq \alpha\}}1dy
\end{align*}
\begin{align*}
	=\int_{-\alpha}^{\alpha} \left(\int_{\RR^{d-1}} 1_{y_2^2+...+y_d^2\leq 1-y_1^2}\,dy_2...dy_d\right)dy_1
\end{align*}
\begin{align*}
=\int_{-\alpha}^{\alpha} \text{Vol}\left(B^{d-1}\left(\sqrt{1-y_1^2}\right)\right)dy_1
\end{align*}
Replacing $\text{Vol}\left(B^{d-1}\left(\sqrt{1-y_1^2}\right)\right)$ with $(\sqrt{1-y_1^2})^{d-1}\text{Vol}\left(B^{d-1}(1)\right)$ since the volume the unit ball with a factor proportional to radius in the power of $d-1$.
\begin{align*}
	=\int_{-\alpha}^{\alpha} (\sqrt{1-y_1^2})^{d-1}\text{Vol}\left(B^{d-1}(1)\right)dy_1
\end{align*}
\begin{align*}
	=\text{Vol}\left(B^{d-1}(1)\right)\int_{-\alpha}^{\alpha} (1-y_1^2)^{(d-1)/2}dy_1
\end{align*}
In the integral part, $\int_{-\alpha}^{\alpha} (1-y_1^2)^{(d-1)/2}dy_1$, notice that $(1-y_1^2)^{(d-1)/2}<1$ in the whole integration domain.
Thus we can write
\begin{align*}
	=\text{Vol}\left(B^{d-1}(1)\right)\int_{-\alpha}^{\alpha} (1-y_1^2)^{(d-1)/2}dy_1
\end{align*}
\begin{align*}
	\leq \text{Vol}\left(B^{d-1}(1)\right)\int_{-\alpha}^{\alpha} 1dy_1
\end{align*}
\begin{align*}
	=2\alpha \text{Vol}\left(B^{d-1}(1)\right)
\end{align*}
Recall that volume of unit ball in d dimensions is asymptotically
\begin{align*}
	V_1 = \frac{1}{\sqrt{d\pi}}\left(\frac{2 \pi e}{d}\right)^{d/2}
\end{align*}
Hence the probability $p$ we are interested in satisfies asymptotically
\begin{align*}
	p=\frac{V_\alpha}{V_1} \propto
\frac{2\alpha\frac{1}{\sqrt{(d-1)\pi}}\left(\frac{2 \pi e}{d-1}\right)^{(d-1)/2}}
 {\frac{1}{\sqrt{d\pi}}\left(\frac{2 \pi e}{d}\right)^{d/2}}
=
\frac{2\alpha\frac{1}{\sqrt{(d-1)\pi}}\left(\frac{2 \pi e}{d-1}\right)^{(d-1)/2}}
{\frac{1}{\sqrt{d\pi}}\left(\frac{2 \pi e}{d}\right)^{(d-1)/2}\left(\frac{2 \pi e}{d}\right)^{1/2}}
\end{align*}
We simplify the last term
\begin{align*}
	=2\alpha* \left(\frac{d}{d-1}\right)^{1/2} *   \left(\frac{d}{d-1}\right)^{(d-1)/2} * \left(\frac{d}{2\pi e}\right)^{1/2}
\end{align*}
\begin{align*}
	=2\alpha*  \left(\frac{d}{\sqrt{2\pi e (d-1)}}\right) *   \left(\frac{d}{d-1}\right)^{(d-1)/2}
\end{align*}
 Since $\frac{d}{d-1} = 1+\frac{1}{d-1}$
\begin{align*}
	=2\alpha*  \left(\frac{d}{\sqrt{2\pi e (d-1)}}\right) *   \left(1+\frac{1}{d-1}\right)^{(d-1)/2}
\end{align*}
We modify the power of the same term, to show it as
\begin{align*}
	=2\alpha*  \left(\frac{d}{\sqrt{2\pi e (d-1)}}\right) *   \left(\left(1+\frac{1}{d-1}\right)^{(d-1)}\right)^{1/2}
\end{align*}
Recall that 
\begin{align*}
	e = \lim_{n \rightarrow \infty} \left(1+1/n\right)^n
\end{align*}
Thus this term is bounded with $\sqrt{e}$
\begin{align*}
	\leq 2\alpha*  \left(\frac{d}{\sqrt{2\pi e (d-1)}}\right) *   \sqrt{e}
\end{align*}
We reformulate as
\begin{align*}
	= \alpha \sqrt{d}\sqrt{\frac{2d}{\pi(d-1)}}
\end{align*}
Since $\sqrt{\frac{d}{d-1}}\leq 2$ for $d\geq 2$
\begin{align*}
	\leq \frac{2\sqrt{2}}{\pi}\alpha \sqrt{d} 
\end{align*}
Given that all of this only holds asymptotically; we might need another multiplicative constant to make it hold in general. Hence the constant $C$ in the theorem.
\begin{align*}
	p \leq  C\alpha \sqrt{d}
\end{align*}
\end{document}

%%%%% section end %%%%%%%%%