% !TeX root = ../notes.tex

\section{Textbook Query Optimization}
Textbook query optimization involves techniques to perform a rough first optimization, which however is quite simple. There is a series of steps to translate raw SQL into logical and physical plans, each of them transforming input in a more optimal form.

The output is going to be executable, but still to be improved by non-trivial methods.

\subsection{Algebra and tuples}
Plain relational algebra is not sufficient itself: it needs to be revisited ensuring correctness (producing the same result) within a formal model. The most relevant problem to tackle is deciding whether two algebraic expressions are the same, but this in difficult in practice.

For instance, performing a selection before a join might be correct (and faster) in case the considered criterion is equality, but can give a different result than selecting after an outer join. 

To remedy this issue, it is possible to guarantee that two expressions are equivalent, not accepting false positing yet allowing false negatives. 

A formal definition of tuple is an unordered mapping from attribute names to values of a domain. A schema consists in a set of attributes with domain $A(t)$.

Tuple operations are:
\begin{itemize}
	\item Concatenation, attaching one tuple to another regardless of ordering (union);
	\item Projection, producing a notation $t.a$ in which it is possible to access single values or multiple $t_{|\{a, b\}}$, getting a subset of the schema.
\end{itemize}

A set of tuples with the same schema forms a relation. Sets naturally do not comply with real data, since they not allow duplicates, but are used for simplicity.

In most cases, sets and bags can be used interchangeably, but the optimizer considers different semantics: logical algebra operates on bags, physical algebra on streams and sets are only considered after an explicit duplicate elimination.

Set operations are the classic ones of union, intersection and difference, yet are subject to schema constraints. On bags, operations are performed on frequencies. 

There are also free variables, which first must be bounded to be evaluated: they are essentials for predicates and algebra expressions, such as dependent joins. 

It is important to note that projection removes duplicates within sets, while keeping them in bags.

There are equivalences for selection and projection useful to derive whether a different ordering produces the same output. For instance, applying selection twice is the same as applying it once with two criteria plus an AND. Commutative property also holds.


